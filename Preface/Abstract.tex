\chapter*{Abstract}
In the multidisciplinary and combined effort to understand the brain, we focus in a bottom up approach, to explain behavioral processes starting from the dynamics at low levels. In this line, we focus on the sequential nature of neuronal dynamics, an essential point of view since many neural processes at different time scales occur in a sequential manner. To explore this sequentiality at different scales we need both adequate case of study, approaches and techniques. In this thesis we have addressed this topic using Central Pattern Generators (CPGs), to explore their sequential and robust pattern generators and a novel neurotechnology for neural dynamics modulation, near-infrared laser stimulation. All over the work we have combined electrophysiology and computational techniques, exploiting the advantages of each one of them. 

First, we examined the presence of strong linear relations in the feeding CPG of the great pond snail (\textit{Lymnaea stagnalis}) characterizing the bursting sequential activation and the emerging time restrictions in the form of sequential dynamical invariants, recently reported in the pyloric CPG of \textit{Carcinus maenas}. We test the universality of this phenomena by exploring it in other system and a modeling study, showing its presence in other system and also discussing its role as indicator for the variability distribution in the system under different contexts. We also discuss the necessity of reproducing intrinsic functional variability in the models for a complete study of the system and this kind of dynamical features. We propose a first approximation to the extrapolation of sequential dynamical invariants into effective locomotion in robotics. 

In the aim of finding novel techniques and approaches to modulate the neural activity, to further test this kind of experiments, we also present here the effect of continuous-wave NIR laser in the neuronal dynamics of single neurons. First, we present its effect on single neurons, proving its validity in sustained deliberation to alter the neuronal dynamics, accelerating the action potentials and the spiking rate. We analyze the different biophysical candidates supported y a theoretical study with computational simulations, exploring both the role of each candidate and the global key role of temperature in the observed effect. To assess the sequential evolution of the spike generation and the effect we design and present a novel activity-dependent technique, to deliver the illumination at specific time instant and dissect the action potential. Besides the potential as research tool, we believe this protocol will be leading in clinical applications, for treatments personalization. 

Overall, this work provides a comprehensive study of neural dynamics, exploring their sequential nature and modulation from the ionic channel level. The identification of dynamical invariants and the non-invasive stimulation through CW-NIR laser, will be key for future applications in neurorrehabilitation, for assisted locomotion and treatments in neural disorders