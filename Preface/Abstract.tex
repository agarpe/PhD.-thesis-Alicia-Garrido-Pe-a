\chapter*{Abstract}
In the multidisciplinary and combined effort to understand the brain, we focus in a bottom-up approach to explain behavioral processes starting from neural dynamics at low description 
 levels. In this line, we study the sequential nature of neuronal dynamics, an essential point of view since many neural processes at different time scales occur in a sequential manner. To explore this sequentiality at different scales we need adequate cases of study, approaches and techniques. In this thesis, we have addressed this topic using neurons and circuits of Central Pattern Generators (CPGs) to explore their robust sequential rhythm generation. and a novel neurotechnology for the noninvaive modulation of neural dynamics: continuous-wave near-infrared laser stimulation. All over the work we have combined electrophysiology and computational techniques, exploiting the advantages of their joint use. 

First, we examined the presence of robust linear relations between intervals that build up the cycle-by-cycle bursting sequence of the feeding CPG of the great pond snail (\textit{Lymnaea stagnalis}) characterizing emerging coordination constraints in the form of sequential dynamical invariants, which were recently reported in the pyloric CPG of \textit{Carcinus maenas}. We assessed the universality of this phenomenon by exploring it in another system and in a modeling study. We found showing its presence in other system and we discuss its role as an indicator of functional variability in the system under different contexts. We also discuss the necessity of reproducing intrinsic functional variability in computational models for a complete characterization of the system and the associated sequential dynamical features. We also propose a first approximation to the extrapolation of sequential dynamical invariants into effective locomotion in biohybrid robotics. 

In the aim of finding novel techniques and approaches to noninvasively modulate sequential neural activity, we also present here an study of the effect of continuous-wave NIR laser in the neuronal dynamics of single neurons. First, we present its effect on single neurons, proving its validity in sustained stimulation to alter neuronal dynamics by accelerating the action potentials and the spiking rate. We analyzed the different biophysical candidates with the help of a theoretical study with computational simulations that explored both the effect of each candidate and the global key role of temperature in the observed modulation. To assess the change in the sequential evolution of the spike generation,  we designed a novel activity-dependent technique, to deliver the illumination at specific time instants. This protocol allowed to dissect the effect of the CW-NIR illumination on the action potential. Besides the potential as a research tool, we believe that this closed loop protocol-will become a widely used neurotechnology in clinical applications, allowing the design of personalized treatments. 

Overall, this work provides a comprehensive study of neural dynamics, exploring their sequential nature and modulation from the ionic channel level. The identification of dynamical invariants and the non-invasive stimulation through CW-NIR laser will be key for future applications in neurorrehabilitation for assisted locomotion and treatments in neural disorders.