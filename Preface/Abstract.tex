\chapter*{Abstract}
In the context of the multidisciplinary and combined efforts to understand the brain, we focus in a bottom-up approach to explain neural dynamics at low description levels. In particular, we study the sequential nature of neuronal dynamics, an essential point of view since many neural processes at different time and spatial scales occur in a sequential manner. To explore this sequentiality at different scales, we need adequate cases of study, approaches and techniques. In this thesis, we have addressed this topic using neurons and circuits of Central Pattern Generators (CPGs) to explore their robust sequential rhythm generation and a novel neurotechnology for the noninvasive modulation of neural dynamics: continuous-wave near-infrared (CW-NIR) laser stimulation. Throughout this work, we have combined electrophysiology and computational techniques, exploiting the advantages of their joint use. 

First, we examined the presence of robust linear relationships between intervals that build up the cycle-by-cycle bursting sequence of the feeding CPG of the great pond snail (\textit{Lymnaea stagnalis}) characterizing emerging coordination constraints in the form of sequential dynamical invariants, which were recently reported in the pyloric CPG of \textit{Carcinus maenas}. We provided evidence to support the universality of this phenomenon by exploring it in another system and in a modeling study. We quantified the variability of the intervals of the sequences and discuss its role in motor coordination under different contexts. We also discuss the necessity of reproducing intrinsic functional variability in computational models for a complete characterization of the system and the associated sequential dynamical features. To conclude this first part of the thesis, we propose the validation of sequential dynamical invariants as flexible autonomous coordination mechanisms for  effective locomotion in biohybrid robotics. 

In the aim of finding novel techniques and approaches to noninvasively modulate sequential neural activity, we present in the second part of the thesis a study of the effect of CW-NIR laser in the neuronal dynamics of single neurons. First, we quantified its action on single neurons proving its effectiveness in sustained stimulation to modulate neuronal dynamics by accelerating action potentials and increasing the spiking rate. We analyzed the different biophysical candidates to explain the results with the help of a theoretical study that explored both the effect of each candidate and the global key role of temperature in the observed modulation. To assess the change in the sequential evolution of the spike generation, we designed a novel activity-dependent technique to deliver the illumination at specific time instants. This protocol allowed to dissect the effect of the CW-NIR illumination on the action potential. Beyond its potential as a research tool, we believe that this activity-dependent protocol could become a widely used neurotechnology in clinical applications, allowing the design of personalized treatments. 

Overall, this work provides a comprehensive study of neural dynamics exploring its sequential nature. The identification of dynamical invariants and the non-invasive stimulation through CW-NIR laser can be key for future applications in neurorrehabilitation, for assisted robotic locomotion technologies,  and for novel treatments in neural disorders.