\chapter*{Resumen}
En el contexto de los esfuerzos multidisciplinares y combinados para entender el cerebro, nos centramos en un enfoque bottom-up para explicar la dinámica neural a bajo nivel de descripción. En particular, estudiamos la naturaleza secuencial de la dinámica neuronal, un punto de vista esencial ya que muchos procesos neurales ocurren de manera secuencial a diferentes escalas temporales y espaciales. Para explorar esta secuencialidad a diferentes escalas, necesitamos casos de estudio, enfoques y técnicas adecuadas. En esta tesis, abordamos este tema utilizando neuronas y circuitos Generadores Centrales de Patrones (CPGs, por sus siglas en inglés) para explorar su robusta generación rítmica secuencial, y una novedosa neurotecnología para la modulación no invasiva de la dinámica neural: la estimulación láser infrarroja continua. A lo largo de este trabajo, combinamos técnicas de electrofisiología y computacionales, explotando las ventajas de su uso conjunto.

En primer lugar, examinamos la presencia de relaciones lineales robustas entre intervalos que componen la secuencia de las ráfagas ciclo a ciclo del CPG alimentario del gran caracol de estanque (\textit{Lymnaea stagnalis}), caracterizando las restricciones emergentes de coordinación en forma de invariantes dinámicos secuenciales, recientemente reportados en el CPG pilórico de \textit{Carcinus maenas}. Evaluamos la universalidad de este fenómeno explorándolo en otro sistema y en un estudio de modelado. Discutimos su papel en la coordinación motora bajo diferentes contextos. También discutimos la necesidad de reproducir la variabilidad funcional intrínseca en modelos computacionales para una caracterización completa del sistema y las características dinámicas secuenciales asociadas. Finalmente, en esta primera parte de la tesis, proponemos la validación de los invariantes dinámicos secuenciales como mecanismos de coordinación dinámica para la locomoción efectiva en la robótica biohíbrida.

Con el objetivo de encontrar nuevas técnicas y enfoques para una modulación no invasiva de la actividad neural secuencial, presentamos en la segunda parte de la tesis un estudio del efecto del láser infrarrojo continuo (CW-NIR) en la dinámica neuronal de neuronas individuales. Primero, mostramos su efecto en estas neuronas, demostrando su eficacia en la estimulación sostenida para modular la dinámica neuronal al acelerar los potenciales de acción y la frecuencia de los potenciales de acción. Analizamos los diferentes candidatos biofísicos para explicar los resultados con la ayuda de un estudio teórico en el que se explora tanto el efecto de cada candidato como el papel clave global de la temperatura en la modulación observada. Para evaluar el cambio en la evolución secuencial de la generación de \textit{spikes}, diseñamos una novedosa técnica dependiente de la actividad para entregar la iluminación en instantes de tiempo específicos. Este protocolo permitió disectar el efecto de la iluminación CW-NIR en el potencial de acción. Más allá de su potencial como herramienta de investigación, creemos que este protocolo de ciclo cerrado puede convertirse en una neurotecnología ampliamente utilizada en aplicaciones clínicas, permitiendo el diseño de tratamientos personalizados.

En resumen, este trabajo proporciona un estudio exhaustivo de la dinámica neural explorando su naturaleza secuencial. La identificación de invariantes dinámicos y la estimulación no invasiva a través de láser CW-NIR serán clave para futuras aplicaciones en neurorrehabilitación para la locomoción asistida y tratamientos en trastornos neurológicos.