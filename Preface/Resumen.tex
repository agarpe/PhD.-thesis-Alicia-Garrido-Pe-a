\chapter*{Resumen}
En el contexto de los esfuerzos multidisciplinares y combinados para entender el cerebro, en esta tesis nos centramos en un enfoque \textit{bottom-up} para explicar la dinámica neuronal desde una descripción de procesos a bajo nivel. En particular, estudiamos la naturaleza secuencial de la dinámica neuronal, un punto de vista esencial ya que muchos procesos neurales ocurren de manera secuencial en diferentes escalas temporales y espaciales. Para explorar esta secuencialidad en diferentes escalas, necesitamos casos de estudio, enfoques y técnicas adecuadas. Por tanto, abordamos este tema utilizando neuronas y circuitos Generadores Centrales de Patrones (CPGs, por sus siglas en inglés) para explorar la generación de sus ritmos secuenciales robustos, y una novedosa neurotecnología para la modulación no invasiva de la dinámica neuronal: la estimulación con láser infrarrojo cercano de onda continua (CW-NIR, por sus siglas en inglés). A lo largo de este trabajo, combinamos técnicas de electrofisiología y modelos computacionales, aprovechando las ventajas de su uso conjunto.

En primer lugar, examinamos la presencia de relaciones lineales robustas entre intervalos que componen la secuencia de las ráfagas ciclo a ciclo del CPG alimentario del gran caracol de estanque (\textit{Lymnaea stagnalis}), caracterizando las restricciones de coordinación emergentes en forma de invariantes dinámicos secuenciales, recientemente reportados en el CPG pilórico de \textit{Carcinus maenas}. Aportamos evidencias que apuntan a la universalidad de este fenómeno explorándolo en otro sistema y en un estudio teórico. Cuantificamos la variabilidad de los intervalos que forman las secuencias y abordamos su papel en la coordinación motora bajo diferentes contextos. También discutimos la necesidad de reproducir la variabilidad funcional intrínseca en modelos computacionales para una caracterización completa del sistema y de las propiedades dinámicas de las secuencias asociadas a esa variabilidad. Finalmente, en esta primera parte de la tesis, proponemos la validación de los invariantes dinámicos secuenciales como mecanismos de coordinación autónoma y flexible para la locomoción efectiva en robótica biohíbrida.

Con el objetivo de encontrar nuevas técnicas y enfoques para una modulación no invasiva de la actividad neural secuencial, presentamos en la segunda parte de la tesis un estudio del efecto del láser infrarrojo continuo (CW-NIR) en la dinámica neuronal de neuronas individuales. Primero, ilustramos su efecto en estas neuronas, demostrando su eficacia mediante iluminación sostenida para modular la dinámica neuronal al acelerar los potenciales de acción e incrementar su frecuencia. Analizamos los diferentes candidatos biofísicos para explicar los resultados con la ayuda de un estudio teórico en el que se explora tanto el efecto de cada candidato como el papel clave de la temperatura en la modulación observada. Para evaluar el cambio en la evolución secuencial de la generación de \textit{spikes}, diseñamos una nueva técnica para iluminar las neuronas solo en instantes de tiempo específicos dependientes de su actividad. Este protocolo permitió diseccionar por etapas el efecto de la iluminación CW-NIR en el potencial de acción. Más allá de su potencial como herramienta de investigación, creemos que este protocolo de ciclo cerrado puede convertirse en una neurotecnología ampliamente utilizada en aplicaciones clínicas, permitiendo el diseño de tratamientos personalizados.

En resumen, este trabajo proporciona un estudio exhaustivo de la dinámica neuronal explorando su naturaleza secuencial. La identificación de invariantes dinámicos y la estimulación no invasiva a través de láser CW-NIR pueden ser claves para futuras aplicaciones en neurorrehabilitación, para el diseño de tecnologías robóticas de locomoción asistida, y para el desarrollo de nuevos tratamientos en trastornos neurológicos.