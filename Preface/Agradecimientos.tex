\chapter*{Agradecimientos}
Casi más dificil que el propio documento es resumir en una página en forma de agradecimientos los esfuerzos de mi entorno durante esta tesis. En este tiempo he aprendido mucho sobre neurociencia, programación, modelos dinámicos, lásers... Pero también mucho a nivel personal, y por supuesto no ha sido sola. Así que a toda persona con la que me he cruzado o he compartido camino y ha vivido esto conmigo, muchas gracias. 

Concretando... Gracias a Pablo Varona, por abrirme las puertas del mundo de la neurociencia y darme espacio y apoyo para desarrollarme en el mundo de la investigación. A Rafi por darle la oportunidad a una informática de aprender electrofisiología y por enseñarme tanto en tantas horas de laboratorio, experimentos y análisis. A Paco por tantas ideas y por el apoyo y cariño durante todo el trabajo de esta tesis. A Jesús y Javier por todas esas horas ajustando el láser. A todos los demás miembros del GNB por acogerme tan bien desde el principio. A Roy por tantas conversaciones recien llegada a "primera hora". A Irene por dedicar tanto tiempo en su final de tesis a enseñarme. A Pablo por tantas horas de experimentos y trabajo, sobretodo por todo el apoyo mutuo en todas esas jornadas. A Sara por el caos que he tenido la suerte de conocer en estos años. A Ángel por tantas ideas, escucha y aportaciones a este trabajo, pero también por los cafés y el deporte para recargar pilas. A Carlos, Alberto, Fabiano, Aaron, Jessica, Manu, Angel F. Habéis hecho más fácil ir cada día al lab y un gusto siempre veros fuera de él. 


Y fuera del lab, no sois pocos los que me habéis apoyado. A Irene, por haber acompañado tan de cerca estos años de perderme las fiestas de Moratalla por el congreso, de desaparecer volviendo a las tantas de trabajar y por haber hecho de los ratos de ocio y COVID en Madrid un espacio de casa, convivencia y cariño. A mi MAPS, por haber sido refugio emocional tantas veces y conseguir encontrarnos siempre pese al tiempo que pase (y la falta de él). A Jorge y Carmen, que también habéis vivido esto muy de cerca, pero sobretodo habéis sido parte de esos ratos de fuga. A Rodri por acompañarme desde el principio de mi etapa en Madrid. A María, por estar siempre en la misma onda vital y comprenderme tan bien. A mi Granada querida y todos los que sé que me apoyáis y mantenéis desde allí (incluso aunque ya no la habitéis). A RadioPalacio y Rumanía, por estarconseguir que el tiempo no consiga cambiar nada al volver a vernos. A Jelen y Ro, por cuidarme tanto en la distancia, gracias por toda la confianza y el cariño con el que me la demostráis. A Cris, o como diría ella mi amiga más antigua, por todo el apoyo que siento de tu parte. A Miguel y Zapa, porque desde que os conozco habéis sido un oásis de escape, desarrollo personal y cable a tierra. A Isa, que siempre consigue estar presente aunque ni la constancia ni el tiempo sean nuestro punto fuerte. A Mai, Const y Anita, por los ratitos breves pero tan casa que hemos compartido estos años.

A mi familia, a mis padres y mis hermanas, por su cuidado y apoyo incondicional durante este proceso y en todos los demás aspectos, retos y cambios de la vida. 

Y por supuesto a Ángel, por su ayuda durante todo este trabajo, porque has sido apoyo indiscutible desde la primera charla online en el congreso del CNS hasta el seminario de fin de tesis. Por ser motor, soporte e impulso constante, y la compañía que quiero en cualquier camino.
