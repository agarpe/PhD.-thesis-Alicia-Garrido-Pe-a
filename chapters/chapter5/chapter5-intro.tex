
Effective neural stimulation is an essential tool to study brain dynamics. Many techniques have risen since the first use of electrical, chemical and mechanical stimulation, e.g., see Refs. \parencite{cogan_neural_2008, chamorro_generalization_2012, carter_guide_2015, bickle_revolutions_2016}. Optical methods are also widely spread, as they allow visualization \parencite{lecoq_wide_2019} and stimulation in a less invasive manner. One example is optogenetics \parencite{boyden_millisecond-timescale_2005, yizhar_optogenetics_2011, tye_optogenetic_2012,bansal_towards_2022}, which is effective in modifying neural activity with high spatio-temporal resolution. Another example of non-invasive stimulation is Transcranial Magnetic Stimulation \parencite{valero-cabre_transcranial_2017}, which is succeeding in clinical applications. However, they both present limitations such as the need to genetically modify the living system or restricted spatial precision, respectively. In this context, infrared laser stimulation is an optical technique that has risen in popularity in the last decade. From its first applications \parencite{wells_application_2005, izzo_optical_2007}, studies have shown its ability for modulating action potentials in different systems \parencite{liang_temperature-dependent_2009, goyal_acute_2012, brown_thermal_2020, barrett_pulsed_2018, shapiro_infrared_2012, cayce_infrared_2014, begeng_activity_2022}. Beyond its potential as a research stimulation technique, it has also been tested for clinical use, e.g., in Parkinson's disease, reversing brain age-related effects or depression treatment \parencite{konstantinovic_transcranial_2013, disner_transcranial_2016, wang_impact_2017, saucedo_transcranial_2021, pan_infrared_2023}. This neural stimulation method is so attractive because of the wide range of possibilities that can provide for non-invasive neuromodulation offering high temporal and spatial precision.

The identification of the biophysical source of infrared neuromodulation is still under discussion as it has strong implications for applications in multiple contexts. It is difficult to associate this modulation to a single specific cause, since neural systems have distinct biophysical components reactive to the irradiation. However, most of the results point to a photo-thermal effect where the excitation driven by the laser stimulation might be caused by temperature gradient \parencite{wells_biophysical_2007}. In addition, different candidates to explain the change in neural activity have been suggested, such as capacitance \parencite{shapiro_infrared_2012, plaksin_thermal_2018}, specific modulation of channels sensitive to temperature as TRPV4 \parencite{albert_trpv4_2012}, acceleration of ionic channels \parencite{liang_temperature-dependent_2009}, or altering the $Ca^{2+}$ cycle possibly mediated by modulation of mitochondrial activity \parencite{dittami_intracellular_2011, lumbreras_pulsed_2014, saucedo_transcranial_2021}.

Distinct types of infrared laser and action modes, in terms of the power, duration, frequency of stimulation and wavelength have been used in previous studies, see Refs. \cite{izzo_optical_2007, wells_application_2005,ping_targeted_2023}. The effect is highly dependent on the stimulation configuration. Most works have focused on pulsed lasers to induce spiking activity due to their stronger temperature gradient production. However, some clinical studies have successfully applied continuous-wave (CW) laser for brain stimulation \parencite{saucedo_transcranial_2021}.

The use of closed-loop techniques has a large potential in neuroscience, for both physiological and clinical research studies \parencite{potter2010, chamorro_generalization_2012, couto_firing_2015,lareo_temporal_2016,varona_online_2016,zrenner_closed-loop_2016,reyes-sanchez_automatic_2020,reyes-sanchez_automatized_2023}, since they allow adjusting the stimulation to the context of the ongoing neural dynamics and the specific condition of the targeted system/subject. Some of these tools have been developed with open-source approaches, including optical techniques, e.g. Refs. \parencite{siegle_neural_2015,dagnew_cerebralux_2017,amaducci_rthybrid_2019,stih_stytra_2019,robbins_optogenie_2021}, promoting the accessibility, reproducibility and standardization of the studies and methods. However, near-infrared (NIR) lasers have been used with fixed/periodic stimulus and, to the best of our knowledge, they have not been exploited in activity-dependent protocols. 

Here we explore the effect of CW-NIR laser on the dynamics of individual neurons in sustained and activity-dependent stimulation protocols. We employ a laser with constant optical power density on the sample for these two modalities. In the first case, the laser stimulation is sustained --the duration of the illumination is constant for more than 1 minute--, and in the second case it is driven in an activity-dependent manner implemented by the open-source RTXI\parencite{patel_hard_2017} Linux software --the onset of the stimulation is determined by ongoing neural events and delivered transiently through software control--.
We studied the effect of CW-NIR illumination focused on neurons with spontaneous tonic firing. Combining experimental results with modeling analysis allowed exploring the candidates that can explain the observed neuromodulation. We present a novel procedure for NIR laser stimulation to dissect and intervene in the waveform dynamics through activity-dependent stimulation. By interlacing results from theoretical simulations and sustained and activity-dependent stimulation, we identify the dynamical elements behind action potential dynamics under CW-NIR modulation. We discard any single candidate of the biophysical effect as the joint experimental and model analyses indicate that laser illumination affects multiple membrane factors simultaneously.
