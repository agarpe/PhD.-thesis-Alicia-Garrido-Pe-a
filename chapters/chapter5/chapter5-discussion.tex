
\subsection{Singularity of the sustained and activity-dependent CW-NIR stimulation on neural dynamics}
Advantages of infrared laser neuromodulation beyond its non-invasive nature include its relative simplicity regarding stimulation protocol design, good penetration depth and the possibility to implement highly selective spatio-temporal stimulation delivery. The effectiveness of future applications will depend on a clear understanding of the mechanisms of the neural dynamics modulation.

Most previous studies used protocols involving high-frequency pulsed lasers under the assumption that continuous-wave laser stimulation paradigms do not provide significant activation or neuromodulation \parencite{wells_application_2005, wells_optical_2005, cayce_infrared_2014, chernov_infrared_2014, goyal_acute_2012, pan_infrared_2023}. Their focus involved heating the neurons to elicit spiking activity. The laser wavelengths used were mostly in the range of 1800nm, close to a water absorption band. Here we explored a different approach, using a 830nm CW-NIR laser in sustained and activity-dependent triggered stimulation instead of pulsed illumination at fixed frequency. This setup has a promising future for clinical applications for long-term stimulation and patient-based treatments.

We assessed the action of sustained and activity-dependent CW-NIR stimulation to unveil the biophysical sources of the observed modulation on neuronal dynamics. We combined experimental and theoretical methods to analyze this effect. First, we quantified the change on action potential waveform dynamics and on the inter-spike intervals by comparing triplets of long intracellular recordings of control, laser stimulation and recovery. We found that sustained exposure to 830nm CW laser effectively modulated the spike waveform in a reversible manner. We showed this modulation in two different neuron types, illustrating the generalization of the effect. We observed a stronger effect on duration and repolarization, followed by a less strong change in the depolarization slope and a minimal change on amplitude. The neuron dynamics was restored after stimulation. It is important to highlight that here we presented modulation of tonic spontaneous activity, not elicitation of spiking activity as in most previous studies \parencite{wells_application_2005,izzo_optical_2007,shapiro_infrared_2012,rabbitt_heat_2016}. We also showed a tendency to increase the spiking activity under sustained stimulation, not limited to a specific time/intensity configuration of laser pulses as it is most frequently done in the literature \parencite{izzo_optical_2007,goyal_acute_2012,beier_plasma_2014,pan_infrared_2023}. Although there are previous studies discussing the inhibitory ability of infrared-laser illumination \parencite{duke_transient_2013,lothet_selective_2017,ganguly_thermal_2019, begeng_activity_2022}, we did not find evidence of any direct CW-NIR inhibitory effect. Note that the origin of tonic spiking was affected by the intrinsic properties of the cell and the synaptic inputs within the circuit, e.g., the illuminated neuron might be triggering an inhibitory or excitatory feedback from other neurons, complicating the analysis. This explains the lack of excitation in a subpopulation in Fig. \ref{fig:frequency FR}. Spontaneous neural activity and the nature of the living preparation used may naturally tend to decrease the firing rate.

\subsection{Biophysical explanation of the CW-NIR modulation through modeling and activity-dependent stimulation}
The results of sustained CW-NIR illumination alone cannot discard previously suggested mechanisms such as cytochrome oxidase \parencite{wang_impact_2017,saucedo_transcranial_2021} or calcium release from internal storage \parencite{lumbreras_pulsed_2014}. The fact that the illumination directly affects the spike waveform but not always translates into an increased firing rate may indicate that there is more than a single mechanism involved. Moreover, our analysis of sustained laser stimulation does not point to a slow change such as the one expected with the liberation of Ca$^{2+}$ caused by a mitochondrial modulation \parencite{dittami_intracellular_2011,lumbreras_pulsed_2014}, since we observed a minimal delay between the illumination onset and the modulatory effect, and the illumination cessation and the recovery. The short exposure in the activity-dependent experiment with quick response time also points to a short timescale effect, such as a direct effect on the ionic channels.

Conductance-based models allowed us to identify the most compatible biophysical explanation to the CW-NIR modulation. We evaluated the capacitance and distinct ionic channels in the parameter space of three conductance-based models, which would be highly costly experimentally. We concluded that all candidates explored contributed to partial reproduction of the waveform modulation but none was sufficient to explain the full observed effect. Capacitance is one of the most discussed candidates \parencite{shapiro_infrared_2012,cayce_infrared_2014,thompson_infrared_2015,plaksin_thermal_2018}. However, in our modeling study, capacitance alone was not able to reproduce the modulation. Although the isolated modification of any channel resulted in a limited explanation of the CW-NIR change, late activation channels such as potassium --preserving the depolarization-repolarization change relation-- or high-activated calcium --necessary for shoulder shape modulation-- seem to play a key role reproducing the observed effect.

Temperature-dependent simulations validated that the best explanation for the sustained laser action is a combined modulation of channels, reproducing the observed change for amplitude, duration and slopes. This supports previous studies' hypothesis that the photo-thermal interaction is key in the NIR laser effect \parencite{wells_application_2005,li_temporal_2013,albert_trpv4_2012,rabbitt_heat_2016, barrett_pulsed_2018,brown_thermal_2020,cury_infrared_2021}. We selectively excluded one channel from the temperature dependency at a time in the CGC-model, which cannot be performed experimentally. We found $I_D$ and $I_{NaP}$ channels to be critical for the activity modulation, radically changing the waveform when altering temperature dependency. Also, canceling the temperature dependency of $I_{HVA}$ largely changed the amplitude, indicating its importance in preserving the observed amplitude-repolarization relation during the modulation.

Finally, a closed-loop protocol allowed altering the action potential at distinct generation phases. We presented a new open-source protocol for spike prediction to stimulate at precise times around the occurrence of the action potentials. The outcome of the CW-NIR effect at distinct time intervals in relation to the timing of the spike's peak highlighted the importance of the stimulus delivery time. By changing the illumination instant, we shifted the effect on the waveform shape, getting different maximum metric changes at different stages of the action potential generation.

These changes in the waveform open a discussion about the biophysical source of this effect. With short closed-loop illumination intervals ($<60ms$), we observed a controlled modulation of neural activity smaller than the effect during the sustained laser illumination. In the open-pipette estimation, the maximum change for the steady state temperature value (1-2ºC) was reached after 1s and the change after 50ms was only of 0.1ºC. This estimation was performed on the preparation's solution, and our laser wavelength is far from the intense water absorption bands. So temperature change could be higher in the neuronal membrane, as the specific heat capacity of the water is $\sim30\%$ larger than the estimated on the membrane \parencite{thompson_modeling_2012}. This would result in a faster temperature increase under heat-inducing stimulation. Also, in the model, we observed a change similar to the experimental results with $\Delta T>=5^{\circ}C$. Thus, the modulation might not be caused by simply heating the surrounding water. In the sustained stimulation, there might be additional modulation sources such as an effect on the mitochondria as it has been discussed in previous studies \parencite{dittami_intracellular_2011, lumbreras_pulsed_2014, saucedo_transcranial_2021}, which we cannot discard as adding to the modulation of ionic channels. However, the effect observed during the activity-dependent stimulation is unlikely to have other than fast sources such as ionic-channels. 
A rigorous characterization of the timescale of the temperature changes induced by the CW-NIR and the associated instantaneous voltage dynamics could provide further insight on the fast and slow biophysical mechanisms underlying the waveform modulation. This is particularly relevant for the design of fast activity-dependent protocols to produce the observed effect safely with minimal biophysical perturbation. An accurate characterization of the relation between temperature and neuronal dynamics under CW-NIR stimulation requires novel highly precise protocols to measure the membrane temperature and fast non-periodic electro-optical shutters controlled by real-time software technology. This could be performed using nanothermomery by estimating the temperature change in the membrane by the emission on nanoparticles in the tissue \parencite{hamraoui_exploring_2023}.

\subsection{Applications for research and clinical use}
The open-source approach described in this thesis can be generalized for any animal and preparation. In addition, our protocol leaves plenty of possibilities for other closed-loop stimulation methodologies, including clinical interventions. We provided an open-access repository with the code to reuse our protocols and the module for RTXI, which can be used with any control hardware including fast electro-optical shutters. 


Regarding the non-invasive nature of the CW-NIR laser effect, we could not observe any damage to the cells linked to the stimulation in our experiments. We can hypothesize that stimulation from a laser with higher power could be tolerated by neurons. The recovery of the neural dynamics after illumination does not mean that the CW-NIR laser stimulation cannot be employed to address laser-driven plasticity in protocols designed for this goal. We have shown that sustained CW-NIR laser effectively accelerates neural dynamics in single neurons affecting a combination of biophysical mechanisms. Also, our results indicate that novel research and clinical applications of the excitability increase of laser stimulation must rely on a careful selection of the stimulus parameters and the timing of the illumination. In this context, the results of our pioneer activity-dependent infrared laser stimulation provide a novel approach to adapt the modulation of neural dynamics to specific applications, particularly in the field of personalized treatments including stimulation-driven plasticity.

