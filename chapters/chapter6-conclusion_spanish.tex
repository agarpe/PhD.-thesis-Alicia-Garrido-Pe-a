%!TEX root = ../_thesis.tex
\chapter{Conclusiones y Discusión} % (fold)

A lo largo de este trabajo, hemos explorado con distintos enfoques experimentales y computacionales complementarios la naturaleza secuencial de la dinámica neuronal: desde la activación secuencial de canales iónicos en la generación de potenciales de acción, hasta la dinámica ciclo a ciclo en circuitos CPG. A continuación, se presenta un resumen de los principales resultados de esta tesis:

\begin{itemize}
	\item Parte 1: Invariantes dinámicos secuenciales.
	\begin{itemize}
		\item Ilustramos la importancia de estudiar dinámicas neuronales secuenciales en diferentes niveles de descripción.
		\item Sugerimos la universalidad de invariantes dinámicos secuenciales al demostrar su presencia en el CPG alimentario de \textit{Lymnaea stagnalis} en datos experimentales y en simulaciones computacionales, en un modelo animal diferente al utilizado en el trabajo que reportó su descubrimiento.
		\item Mostramos que los invariantes dinámicos pueden identificarse en modelos cuando el circuito exhibe suficiente variabilidad ciclo a ciclo inducida por estímulos externos.
		\item Los modelos que generaban variabilidad mediante la estimulación neuronal predijeron que los invariantes cambian según la neurona estimulada. Dicha predicción se comprobó experimentalmente.
		\item Los invariantes dinámicos sirven como indicadores de la distribución de variabilidad funcional en el CPG.
		\item La variabilidad neuronal intrínseca en los modelos comúnmente utilizados es limitada comparada con la observada experimentalmente, sin embargo, es esencial para estudiar la dinámica neuronal secuencial.
		\item Demostramos el papel funcional de los invariantes dinámicos secuenciales mediante el uso de relaciones lineales robustas entre  intervalos temporales de la actividad neuronal para controlar la coordinación motora en un robot biohíbrido.
	\end{itemize}
	
	\item Parte 2: Iluminación con láser CW-NIR como técnica de modulación neuronal efectiva.
	\begin{itemize}
		\item La iluminación sostenida con láser CW-NIR acelera asimétricamente la dinámica de los potenciales de acción y aumenta la frecuencia de disparo en neuronas individuales.
		\item Los resultados preliminares en neuronas eléctricamente acopladas (como circuito mí\-nimo) muestran el potencial del láser CW-NIR para modular la dinámica en circuitos.
		\item Mostramos que distintas longitudes de onda del láser CW-NIR pueden evocar diferentes cambios en las formas de onda de los potenciales de acción, resaltando la importancia de seleccionar la longitud de onda, la potencia y la duración para objetivos específicos de modulación neuronal.
		\item    El estudio computacional del efecto del láser CW-NIR reveló que ningún único candidato biofísico puede replicar completamente la modulación observada y la explicación más plausible es la modulación global a través del cambio de la temperatura.
		\item Con el protocolo de ciclo cerrado, mostramos el efecto del láser CW-NIR en varias fases de la dinámica neuronal de la generación del potencial de acción, demostrando cambios en la modulación neuronal según el momento de la iluminación.
		\item El protocolo de ciclo cerrado disponible en código abierto puede generalizarse fácilmente para otro tipo de neuronas y contextos de investigación.
	\end{itemize}
	
\end{itemize}

En el capítulo \ref{c-invariants}, analizamos los invariantes dinámicos secuenciales en un circuito CPG. El objetivo fue demostrar la hipótesis presentada en \textcite{elices_robust_2019} de que las relaciones lineales robustas encontradas entre intervalos ciclo a ciclo en el CPG pilórico del \textit{Carcinus maenas} no eran una característica específica de ese CPG en particular, sino un fenómeno general para la coordinación motora autónoma. El principal resultado del capítulo \ref{c-invariants} fue mostrar \textbf{la generalización de la presencia de los invariantes dinámicos secuenciales}, ya que se encontraron no solo en un modelo detallado del circuito, sino también en registros experimentales del CPG alimentario de \textit{Lymnaea stagnalis}. En el modelo, cuantificamos la variabilidad inducida por una corriente en forma de rampa en neuronas específicas. Exploramos la variabilidad y las correlaciones entre el período y los diferentes intervalos en el ciclo, mostrando la presencia de invariantes dinámicos secuenciales. Observamos que \textbf{la distribución de variabilidad y, por lo tanto, los invariantes dinámicos secuenciales cambian según la neurona estimulada en la simulación del modelo}. Esto resalta la importancia del balance entre robustez y flexibilidad para la coordinación motora mediante la relación entre las neuronas en la dinámica del circuito, de modo que la variabilidad restringida permite una adaptación autónoma ciclo a ciclo. Por lo tanto, la configuración del modelo computacional y la dinámica resultante deben alcanzar un grado de variabilidad mínima en las simulaciones para garantizar la existencia de los invariantes dinámicos. Apoyamos estos resultados analizando diferentes registros de neuronas en el ganglio bucal. Discutimos en ese capítulo las dificultades para definir las tres fases del CPG alimentario en comparación con el CPG pilórico. Después de identificar las tres fases del circuito mediante la combinación de señales de interneuronas y motoneuronas, analizamos la variabilidad y las relaciones entre los intervalos ciclo a ciclo para casos de actividad espontánea y actividad inducida por estimulación eléctrica en neuronas individuales. Caracterizamos la variabilidad de intervalos y sus relaciones ciclo a ciclo en diferentes registros de actividad sin estímulo y \textbf{encontramos invariantes dinámicos secuenciales en la actividad neuronal espontánea}. También analizamos diferentes casos de estimulación, y reprodujimos los resultados predichos por el modelo, \textbf{mostrando un cambio en la distribución de variabilidad cuando el ritmo es modulado por SO}. A partir de la premisa de que el ritmo es diferente según varía su fuente de activación, analizamos casos donde el ritmo fue modulado por la estimulación de la neurona CV1a y por el nervio MLN. En ambos casos, \textbf{la variabilidad de los intervalos se distribuyó entre las fases N1 y N3}.

Motivados por estos resultados experimentales y teóricos, también presentamos en el capítulo \ref{c-invariants} un estudio sobre la variabilidad neuronal. En los modelos, \textbf{analizamos la importancia de la variabilidad en la dinámica secuencial neuronal y las limitaciones de los modelos clásicos para reproducir la variabilidad funcional intrínseca de las neuronas}. Comparamos ejemplos de actividad en neuronas vivas y modeladas y su variabilidad, \textbf{analizando diferentes modelos y sus limitaciones}. En cuanto al enfoque experimental, exploramos la posibilidad de emplear los intervalos secuenciales de invariantes dinámicos para coordinar eficazmente la locomoción de un robot biohíbrido. Esto es importante para abordar el papel funcional de estas relaciones robustas que observamos en intervalos específicos de secuencias neuronales, que como discutimos, están influenciadas por el contexto y el origen del ritmo. En este primer prototipo, \textbf{demostramos que es posible lograr un movimiento efectivo del robot biohíbrido manteniendo estas relaciones lineales ciclo a ciclo}.

En el estudio de la secuencialidad en los CPGs, vimos la importancia de utilizar herramientas eficaces para alterar la actividad neuronal de forma no invasiva. En el capítulo \ref{c-laser}, \textbf{exploramos en detalle el láser CW-NIR como técnica de estimulación de forma experimental, teórica y en protocolos dependientes de la actividad}. Primero, mostramos el efecto de iluminar células individuales y cómo \textbf{la iluminación sostenida acelera asimétricamente la dinámica del potencial de acción y aumenta la frecuencia de  \textit{spikes} en neuronas individuales}. Probamos el efecto del láser en un circuito mínimo \textbf{iluminando neuronas acopladas eléctricamente, mostrando que la principal modulación fue en la neurona iluminada, modificando ligeramente la neurona acoplada} que no recibía la fuente de luz. También mostramos resultados preliminares de la relación efecto-longitud de onda, analizando la acción del láser en duración, amplitud y pendientes del potencial de acción con diferentes longitudes de onda y potencias, \textbf{mostrando una modulación de la amplitud a medida que aumenta la longitud de onda y la potencia del láser que no se observa en valores de baja longitud de onda}. El efecto observado experimentalmente en la forma de onda se diseccionó mediante simulaciones en modelos de conductancia, explorando los posibles candidatos para explicarlo: canales iónicos y capacitancia, con y sin descripción de dependencia de temperatura, mostrando que \textbf{ningún candidato por sí solo reproduce completamente la modulación observada y que la modulación global a través de la descripción del cambio de temperatura es la explicación más plausible}. Finalmente, \textbf{presentamos un protocolo en ciclo cerrado para alterar el potencial de acción en distintas fases de su generación secuencial que puede generalizarse para distintos sujetos y neuronas}. Con este protocolo de estimulación, \textbf{caracterizamos el efecto del láser CW en diferentes fases de la dinámica neuronal, identificando diferentes cambios en la onda según el momento de iluminación}. También respaldamos estos resultados con \textbf{una simulación del modelo, ajustando algunos de los canales solo en las fases de despolarización y repolarización, mostrando el aumento del efecto a medida que cambia el tiempo de estimulación}.

El estudio de las restricciones temporales en la activación secuencial de circuitos motores puede tener fuertes implicaciones en neurorrehabilitación y en la comprensión del procesamiento cerebral de la activación motora. Este análisis puede revelar aspectos clave de la relación entre la variabilidad neuronal, la secuencialidad robusta y la flexibilidad en el control motor que se observa en sistemas vivos, como en los CPG. En este contexto, es necesario ampliar el estudio de la variabilidad neuronal secuencial en diferentes animales, pero también relacionar los invariantes dinámicos secuenciales con su papel en la acción motora, asociando su activación en diferentes contextos. Estas restricciones pueden ser indicadores del estado del sistema y pueden asociarse a la importancia de modular cada fase en el momento actual, por ejemplo, la fase de protracción en presencia de alimentos es crucial para la ingesta, como en el ejemplo de estimulación MLN, mientras que su papel cuando el circuito está activo en ausencia de alimentos no es tan importante, por lo que la variabilidad puede distribuirse de manera diferente, como en el caso de la actividad espontánea con una alta correlación de N3 con el período. En esta dirección, es importante mejorar la variabilidad intrínseca del modelo con una descripción ampliada de la dinámica caótica de la actividad neuronal, y su expansión a circuitos y sistemas más complejos. El estudio de las restricciones entre intervalos de tiempo secuenciales en circuitos vivos que producen y mantienen los invariantes dinámicos también puede trasladarse al diseño de robots que reproduzcan la robustez de la coordinación observada en los CPGs. Estos estudios son de gran interés para aplicaciones en neurotecnología y el diseño de robótica autónoma.

La estimulación láser CW-NIR puede ayudar a explorar más en detalle la dinámica neuronal secuencial me\-diante su modulación efectiva. El estudio de esta nueva técnica neuromoduladora puede tener fuertes implicaciones tanto para investigación como para aplicaciones clínicas debido a su naturaleza no invasiva. La estimulación láser ha ido ganando terreno en el campo de la estimulación óptica, a menudo implementada como estimulación con láseres pulsados de longitud de onda larga. Es importante validar diferentes opciones de estimulación como la estimulación sostenida del láser CW-NIR y la esti\-mulación dependiente de la actividad, de gran interés para tratamientos personalizados. El protocolo presentado en este trabajo podría mejorarse con una mayor precisión temporal, llegando incluso a la capacidad de estimular en escala de nanosegundos. Esto permitirá extrapolarlo a dife\-rentes sistemas, independientemente de su escala temporal de activación neuronal, y también para di\-seccionar con precisión la dinámica de activación de los canales iónicos. Además, una caracterización en detalle de la iluminación láser y su rango de seguridad de actuación será de gran importancia para el desarrollo de futuras aplicaciones clínicas. En este trabajo hemos mostrado un efecto notable y reversible que no dañó las neuronas iluminadas, lo que indica que la potencia utilizada podría aumentarse aún más. Una caracterización precisa de la potencia tolerada por las neuronas establecería los rangos de seguridad de su aplicación. El estudio preliminar de la relación efecto-longitud de onda mostró que no solo es importante la potencia, sino la longitud de onda del láser en términos del efecto observado, por lo que el análisis de la relación de la longitud de onda podría ampliar el rango de aplicaciones y ayudar a distinguir las causas del efecto entre otros posibles mecanismos como los fotoeléctricos o fototérmicos. Una línea de estudio futura también podría evaluar el efecto del láser CW-NIR en circuitos, según los resultados preliminares obtenidos, ya que debería ser posible modular la actividad de un circuito globalmente o mediante la estimulación de componentes indivi\-duales. Esto podría estudiarse en los circuitos  CPG, como el caso del CPG alimentario de \textit{L. stagnalis}, iluminando neuronas que tienen un papel activador/modulador en el circuito, como la neurona CGC en el ganglio cerebral, alejada del bucal pero conectada a él. Finalmente, como se discutió en este trabajo y en la literatura sobre neurotecnología láser, el efecto fototérmico es uno de los principales responsables de la modulación neuronal resultante, sin embargo, con las herramientas y técnicas actuales no es posible tener una descripción detallada del cambio de temperatura en la célula. Aunque los resultados preliminares no se han incluido en este trabajo, hemos explorado la posibilidad de una estimación más precisa del cambio de temperatura utilizando nanopartículas de plata. Estas partículas podrían alojarse directamente en la membrana, reduciendo el área de medición del cambio de temperatura, y la variación de la temperatura podría estimarse a partir de su irradiación. Esta medición precisa de la temperatura también contribuiría a mejorar las simulaciones del modelo, con el dato específico del cambio de temperatura en las neuronas. También permitiría explorar el cambio rápido observado en el protocolo dependiente de la actividad que se produce con un pequeño cambio en la temperatura (según la estimación de la pipeta) y aportar información adicional para comprender la fuente del efecto.

En esta tesis hemos utilizado el potencial de una aproximación multidisciplinar en el estudio de la dinámica neuronal secuencial y de técnicas eficaces para modularla. Combinando metodologías experimentales y computacionales, hemos podido proporcionar un análisis detallado de la dinámica secuencial en circuitos CPG y explorar el efecto modulador de una nueva neurotecnología basada en láser CW-NIR en la  dinámica de generación de potenciales de acción. La combinación de metodologías ha demostrado las amplias posibilidades que proporcionan los protocolos de ciclo cerrado en tiempo real para proporcionar una modulación secuencial eficaz. El estudio de secuencias neuronales en múltiples escalas espaciales y temporales puede aportar nuevas ideas para relacionar la dinámica neuronal y el comportamiento. Las nuevas neurotecnologías contribuirán a identificar y explotar la naturaleza secuencial del procesamiento de información neuronal en múltiples sistemas nerviosos. Los resultados discutidos en esta tesis en el contexto de secuencias neuronales también pueden tener fuertes implicaciones en el campo de la neurorrehabilitación, la robótica y la inteligencia artificial por las nuevas perspectivas derivadas de la comprensión de los invariantes dinámicos secuenciales.