%!TEX root = ../_thesis.tex
\chapter{Conclusiones y Discusión} % (fold)

Durante este trabajo, hemos explorado con distintos enfoques experimentales y computacionales complementarios la naturaleza secuencial de la dinámica neuronal, desde la activación secuencial de canales iónicos en la generación de potenciales de acción hasta la dinámica ciclo a ciclo en circuitos de CPG. Aquí se presenta un resumen de los principales resultados presentados en esta tesis:

\begin{itemize}
	\item Parte 1: Invariantes dinámicos secuenciales.
	\begin{itemize}
		\item Ilustramos la importancia de estudiar dinámicas secuenciales en diferentes niveles de descripción.
		\item Sugerimos la universalidad de invariantes dinámicos secuenciales al demostrar su presencia en el CPG alimentario de \textit{Lymnaea stagnalis} en datos experimentales y de modelado, es decir, en un modelo animal diferente al utilizado en el trabajo original que reportó su descubrimiento.
		\item Mostramos que los invariantes dinámicos pueden identificarse en modelos cuando el circuito exhibe suficiente variabilidad ciclo a ciclo inducida por estímulos externos.
		\item La distribución de variabilidad en los intervalos que constituyen la secuencia dentro del CPG alimentario cambia según el neurón estimulado en simulaciones del modelo.
		\item Reprodujimos experimentalmente los hallazgos de invariantes dinámicos secuenciales observados en simulaciones de modelos computacionales.
		\item Los invariantes dinámicos sirven como indicadores de la distribución de variabilidad funcional en el CPG.
		\item La variabilidad neural intrínseca en los modelos suele ser limitada pero crucial para estudiar la variabilidad secuencial en la dinámica.
		\item Demostramos el papel funcional de los invariantes dinámicos secuenciales mediante relaciones lineales robustas de intervalos neuronales para controlar la coordinación motora en un robot híbrido.
	\end{itemize}
	
	css
	
	\item Parte 2: Iluminación láser CW-NIR como técnica moduladora efectiva.
	\begin{itemize}
		\item La iluminación sostenida con láser CW-NIR acelera asimétricamente la dinámica de potenciales de acción y las tasas de disparo en neuronas individuales.
		\item Resultados preliminares en neuronas eléctricamente acopladas (un circuito mínimo) muestran el potencial de CW-NIR para modular la dinámica del circuito.
		\item Variar las longitudes de onda del láser CW-NIR puede evocar diferentes cambios en las métricas de potenciales de acción, ilustrando la importancia de seleccionar longitud de onda, potencia y duración para objetivos específicos de modulación.
		\item Un estudio del modelo de los efectos de CW-NIR reveló que ningún candidato biofísico único podría replicar completamente la modulación observada; en cambio, una modulación global a través de la descripción de la temperatura fue la aproximación más cercana.
		\item Introdujimos un protocolo de lazo cerrado para modular los potenciales de acción en fases específicas de su generación, que puede generalizarse entre diferentes sujetos y neuronas.
		\item El protocolo de lazo cerrado reveló el efecto del láser CW en varias fases de la dinámica neuronal, demostrando diferentes modulaciones según el momento de la iluminación.
	\end{itemize}
	
\end{itemize}
En el capítulo \ref{c-invariants}, analizamos los invariantes dinámicos secuenciales en un circuito CPG. El objetivo fue demostrar la hipótesis presentada en \cite{elices_robust_2019} de que las fuertes relaciones lineales encontradas entre algunos intervalos ciclo a ciclo en el CPG pilórico del \textit{Carcinus maenas} no eran una característica específica de ese CPG en particular, sino un fenómeno general para la coordinación motora autónoma efectiva. Así, el primer resultado mostrado en \ref{c-invariants} fue \textbf{la universalidad de los invariantes dinámicos secuenciales}, encontrados también en el CPG de alimentación de \textit{Lymnaea stagnalis}. Lo demostramos mostrando su presencia no solo en un modelo detallado del circuito, sino también en registros experimentales del CPG. En el modelo, cuantificamos la variabilidad inducida por una corriente rampa en neuronas específicas. Exploramos la variabilidad y las correlaciones entre el período y los diferentes intervalos en el ciclo, mostrando la presencia de invariantes dinámicos secuenciales. \textbf{La distribución de variabilidad y, por lo tanto, los invariantes dinámicos secuenciales cambian según la neurona estimulada en la simulación del modelo}. Esto resalta la importancia de la relación entre las neuronas en la dinámica del circuito, de modo que la variabilidad restringida resulta en flexibilidad para una adaptación ciclo a ciclo. Por lo tanto, la configuración del modelo computacional y la dinámica resultante deben ser lo suficientemente flexibles para lograr esta variabilidad restringida en las simulaciones. Apoyamos estos resultados analizando diferentes registros de neuronas en el ganglio bucal, que representan el circuito. Discutimos en ese capítulo las dificultades para definir las tres fases del CPG de alimentación en comparación con el CPG pilórico. Así, después de identificar las tres fases del circuito basadas en la combinación de señales de interneuronas y motoneuronas, analizamos la variabilidad y las relaciones entre los intervalos ciclo a ciclo para casos de actividad espontánea y actividad impulsada por estimulación actual en registros electrofisiológicos. Mostramos la variabilidad de intervalos y relaciones en tres ejemplos diferentes de actividad espontánea y \textbf{encontramos invariantes dinámicos secuenciales en la actividad neural espontánea en vivo como en el modelo}. También analizamos diferentes casos de estimulación, primero estimulación impulsada por SO, espontáneamente y por estimulación inducida, y reproducimos los resultados en el modelo, \textbf{mostrando un cambio en la distribución de variabilidad cuando el ritmo es modulado por SO}. A partir de esta premisa de que el ritmo es diferente según varía la fuente, analizamos casos donde el ritmo fue modulado por estimulación de la neurona CV1a y por nervio MLN. En ambos casos, \textbf{la variabilidad de los intervalos se desplazó principalmente entre las fases N1 y N3 bajo los diferentes escenarios de estimulación, aunque la tendencia en CV1a no es tan clara como bajo la estimulación SO}.

Basados en estos resultados experimentales y teóricos, también presentamos en \ref{c-invariants} un estudio sobre la variabilidad neural. En el lado del modelado, \textbf{analizamos la importancia de la variabilidad en el estudio de la dinámica neuronal y las limitaciones de la mayoría de los modelos para reproducir la variabilidad funcional intrínseca de las neuronas}. Comparamos diferentes ejemplos de neuronas vivas y modeladas y su variabilidad, \textbf{mostrando diferentes modelos, sus limitaciones y los mejores candidatos para reproducir la variabilidad funcional mediante la descripción de la dinámica}. En cuanto al enfoque experimental, exploramos la posibilidad de transformar los intervalos secuenciales en movimiento efectivo de un robot. Esto es importante para explicar el significado funcional de estas fuertes relaciones que observamos en intervalos específicos, que como discutimos, están relacionadas con el contexto y la fuente del ritmo. En este primer prototipo, \textbf{analizamos y observamos que es posible lograr un movimiento efectivo de un robot manteniendo las fuertes relaciones lineales}.

En el estudio de la secuencialidad en los CPGs, vimos la importancia de utilizar herramientas efectivas para alterar la actividad neural espontánea y reproducir mecanismos vivos o lograr nuevos comportamientos. En el capítulo \ref{c-laser}, \textbf{exploramos en detalle el láser CW-NIR como técnica de estimulación experimental, teórica y en protocolos dependientes de la actividad}. Primero, mostramos el efecto de iluminar células individuales y cómo \textbf{la iluminación sostenida asimétricamente acelera la dinámica del potencial de acción y la tasa de espigas en neuronas individuales}. Probamos la sensibilidad de la estimulación al enfoque al \textbf{iluminar neuronas acopladas eléctricamente, mostrando que la principal modulación fue en la neurona directamente iluminada, modificando ligeramente la neurona acoplada}. También mostramos los resultados preliminares de la relación efecto-longitud de onda, analizando el efecto en duración, amplitud y pendientes con diferentes longitudes de onda y potencias láser, \textbf{mostrando una modulación de la amplitud a medida que aumenta la longitud de onda y la potencia del láser que no se muestra en valores de baja longitud de onda}. Este efecto observado experimentalmente fue disecado a través de simulaciones del modelo, explorando los posibles candidatos: canales iónicos y capacitancia, en modelos basados en conductancia, con y sin descripción de dependencia de temperatura, mostrando que \textbf{ningún candidato por sí solo pudo reproducir completamente la modulación observada y que la modulación global a través de la descripción de temperatura fue la aproximación más cercana a ella}. Finalmente, \textbf{presentamos un protocolo en bucle cerrado para alterar el potencial de acción en distintas fases generacionales secuenciales que puede generalizarse para distintos sujetos y neuronas}. Con este protocolo de estimulación, \textbf{revelamos el efecto del láser CW en diferentes fases de la dinámica neuronal, desplazando el efecto máximo en diferentes momentos de generación de espiga}. También respaldamos estos resultados con \textbf{una simulación del modelo, ajustando algunos de los canales solo en las fases de despolarización y repolarización, mostrando el aumento del efecto a medida que cambia el tiempo de enfoque}.

El estudio de las restricciones temporales en la activación secuencial de circuitos motores puede tener fuertes implicaciones en neurorrehabilitación y en la comprensión del procesamiento cerebral de la activación motora. Esto puede revelar los aspectos clave de la relación entre variabilidad, secuencialidad robusta y flexibilidad que se muestra en sistemas vivos, como en los CPG. Para esto, es necesario ampliar el estudio de la variabilidad secuencial en diferentes animales, pero también relacionar los invariantes dinámicos secuenciales con su papel en la salida motora, relacionando su activación con diferentes contextos. Estas restricciones pueden ser indicadores del estado del sistema y pueden asociarse a la importancia de cada fase en el momento actual, por ejemplo, la fase de protracción en presencia de alimentos es crucial para obtener ingresos alimentarios, como en el ejemplo de estimulación MLN, mientras que su papel cuando el circuito está activo en ausencia de alimentos no es tan importante, por lo que la variabilidad puede distribuirse de manera diferente, como fue el caso del ejemplo espontáneo con una alta correlación de N3 con el período. En esta dirección, es importante mejorar las simulaciones del modelo tanto para la flexibilidad y las relaciones del circuito como para la descripción de la dinámica caótica de la actividad neural, y su expansión a circuitos y sistemas más complejos. El estudio de las restricciones de intervalo de tiempo en circuitos vivos que producen y aseguran los invariantes dinámicos secuenciales también puede transformarse en ejemplos visuales efectivos como el movimiento de robots, reproduciendo la robustez de las secuencias de CPG. Esto será de gran interés para la neurotecnología y el diseño de robótica.

La estimulación láser puede ayudar a explorar aún más esta dinámica mediante la modulación efectiva. El estudio de esta nueva técnica neuromoduladora puede tener fuertes implicaciones tanto para la investigación como para las aplicaciones clínicas debido a su naturaleza no invasiva. La estimulación láser ha ido ganando terreno en el campo de la estimulación óptica, a menudo estudiada como estimulación láser pulsada de longitud de onda larga. Es importante validar diferentes opciones de estimulación como la continua sostenida CW-NIR presentada aquí, pero también la estimulación dependiente de la actividad, de gran interés para tratamientos personalizados. El protocolo presentado en este trabajo podría mejorarse con una mayor precisión, siendo capaz de estimular a escala de nanosegundos. Esto será relevante para extrapolarlo a diferentes sistemas, independientemente de su escala temporal de activación neural, pero también para diseccionar precisamente la dinámica de activación de los canales. Además, una caracterización profunda de la iluminación láser y su rango de seguridad de activación será de gran importancia para las aplicaciones clínicas. En este trabajo, mostramos un efecto fuerte y reversible que no dañó las neuronas iluminadas, lo que indica que el poder utilizado podría aumentarse. Una caracterización precisa del poder tolerado por las neuronas establecería los rangos de seguridad de su aplicación. Además, el estudio preliminar de la relación efecto-longitud de onda mostró que no solo es importante la potencia, sino la longitud de onda del láser en términos del efecto observado, por lo que el análisis de la relación de la longitud de onda podría ampliar este rango de aplicaciones pero también distinguir entre otros mecanismos diferentes como fotoeléctrico, fototérmico, etc. Una línea de estudio futura también evaluaría el efecto del láser CW-NIR en circuitos, según los resultados preliminares, debería ser posible modificar las neuronas en un circuito mediante la estimulación de una de sus neuronas. Esto podría estudiarse en los circuitos de CPG, en el caso del CPG de alimentación, iluminando neuronas que tienen un papel activador/modulador en el circuito, como la neurona CGC en el ganglio cerebral, alejada del bucal pero conectada. Finalmente, como se discutió en este trabajo y en la literatura sobre neurotecnología láser, un efecto fototérmico es uno de los principales actuadores detrás de la estimulación láser neuronal, sin embargo, con las herramientas y técnicas actuales no es posible tener una descripción detallada del cambio de temperatura en la célula. En esta línea, aunque no está incluido en este trabajo, hemos explorado la posibilidad de una estimación más precisa del cambio de temperatura utilizando nanopartículas de plata. Estas partículas podrían estar directamente en la membrana, reduciendo el área de medición del cambio de temperatura, y la variación de la temperatura podría estimarse mediante su irradiación. Esta medición precisa de la temperatura también podría ayudar a mejorar las simulaciones del modelo, con el cambio específico en las neuronas. También explorar el cambio rápido observado en el protocolo dependiente de la actividad con un pequeño cambio en la temperatura según la estimación de la pipeta, puede ayudar a comprender la fuente del efecto.

El estudio de secuencias neuronales en múltiples escalas espaciales y temporales puede proporcionar nuevas ideas para relacionar la dinámica neural y el comportamiento. Las nuevas neurotecnologías pueden contribuir a identificar y explotar la naturaleza secuencial del procesamiento de información neural. Los resultados discutidos en esta tesis en el contexto de secuencias neuronales también pueden tener fuertes implicaciones en el campo de la neurorrehabilitación, la robótica y la inteligencia artificial.