\chapter{Motivation and Objectives}\label{c-review}
During this work, we will analyze and explore neuronal activity focusing on its dynamics and following a bottom-up approach. In the state of art we discussed the importance of sequential dynamics, and how we can describe this sequentiality at different scales, from milliseconds (ionic channels) up to hours or days in terms of behavior or cycles. It is important to study neuronal dynamics at different levels to understand the mechanisms behind the information processing and the subsequent outputs. To explore sequential activation at cell levels, we will use intracellular recordings of single neurons and cells from the feeding CPG circuit of \textit{Lymnaea stagnalis} combining an experimental study with conductance-based models. In the case of the CPG sequential activity, we will characterize and quantify cycle-by-cycle the variability of the sequence intervals. Although this approach has not always been typical in these kinds of studies \parencite{anwar_interanimal_2022}, the assessment of the variability of sequence time-intervals can unveil important characteristics and constraints for the motor coordination \parencite{elices_robust_2019}. Bursting activity is a good case study for this, since it is usually easier to define the phases associated to the context based on the burst events. In this work, we will discuss the generalization of the presence of sequential dynamical invariants, discovered in the pyloric CPG \parencite{elices_robust_2019}, exploring them in the feeding CPG of \textit{Lymnaea stagnalis}, both in a computational model and in experimental recordings. We will explore the restrictions underlying the CPG but also the possible functional role of these restrictions under different cases of activation of the CPG, e.g., feeding initiated by the presence of food or feeding activity initiated even in the absence of food. 

On the other hand, we also reviewed in the previous chapter the importance of stimulation techniques to explore the neural dynamics and behavior but also for possible clinical applications. In the second part of this work, we will explore Continuous-Wave Near-Infrared (CW-NIR) laser, a novel technique for neuronal dynamics modulation. This optical stimulation has potential for effective modulation of neuronal dynamics, but its exact mode of action is not known yet. There are different hypotheses, each one pointing to different candidates being affected during signal modulation in the process of the signal modulation, such as calcium channels, capacitance or long-term process as mitochondria modulation, as well as different sources for the effect, as photo-thermal or photo-chemical effect. During this work we will characterize its effect in single neurons combining experimental and theoretical approaches to narrow down the possible candidates and the source of the effect. We will also present a novel approach for closed-loop stimulation to predict spikes and stimulate at different stages of the action potential generation. 

In summary the objectives of this thesis are:
\begin{enumerate}
    \item To explore the sequential nature of neuronal dynamics. 
    \item To analyze the feeding CPG of \textit{Lymnaea stagnalis} to provide evidence of the universality of sequential dynamical invariants found in the pyloric CPG.
    \begin{enumerate}
        \item To characterize the sequential invariants cycle-by-cycle in the bursting model. 
        \item To analyze intracellular recordings with spontaneous activity and with different rhythm initiation stimulations. 
    \end{enumerate}
    \item To study the restrictions of variability in the reproduction of neural models and circuits.
    \item Discuss the possible functionality of sequential dynamical invariants in robotics. 
    \item Test the capability of CW-NIR laser stimulation to modify neural activity. 
    \begin{enumerate}
        \item Characterization of the effect in the spike waveform dynamics. 
        \item Analysis of the ability to change spiking activity and circuits dynamics. 
    \end{enumerate}
    \item Study the possible candidates underlying the effect in model simulation. 
    \item Design and implement a new technique for the stimulation in closed-loop. 
\end{enumerate}

Understanding the sequential patterns at the level of CPGs and single neurons allows us to identify universal principles that can be applied to broader neural circuits and behaviors. The sequential dynamics explored in the first part of this thesis provide foundational insights into the intrinsic timing mechanisms that govern neural processes. By exploring novel modulator techniques, as near-infrared laser stimulation, we bridge the gap between neural dynamics and the necessary modulation techniques. Also, will allow the precise dissection of sequential activation at ionic-channels by the activity-dependent laser illumination. 
