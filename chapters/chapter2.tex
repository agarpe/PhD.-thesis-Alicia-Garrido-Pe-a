\chapter{Motivation and Objectives}\label{c-review}
During this work, we will analyze and explore neuronal activity focusing on its dynamics and following a bottom-up approach. In the state of art, we discussed the importance of sequential dynamics, and how we can describe this sequentiality at different scales, from milliseconds (ionic channels) up to hours or days in terms of behavior or life cycles. It is important to study neuronal dynamics at different spatial levels to understand the mechanisms behind the information processing and the subsequent outputs. To explore sequential activation at cell levels, we will use intracellular recordings of single neurons and cells from CPG circuits combining an experimental study with conductance-based models. In the case of CPG sequential activity, we will characterize and quantify cycle-by-cycle the variability of the sequence intervals. Although this approach has not always been typical in this kind of studies \parencite{anwar_interanimal_2022}, the assessment of the variability of sequence time-intervals can unveil important characteristics and constraints for motor coordination \parencite{elices_robust_2019}. Bursting activity is a good case study for this goal, since it is usually easier to define the phases associated to the context based on burst events. In this work, we will discuss the generalization of the presence of sequential dynamical invariants, discovered in the pyloric CPG \parencite{elices_robust_2019}, exploring them in the feeding CPG of \textit{Lymnaea stagnalis}, both in a computational model and in experimental recordings. We will explore the constraints underlying the CPG sequence time intervals but also the possible functional role of these flexible restrictions under different cases of activation of the CPG, e.g., feeding initiated by the presence of food or feeding activity initiated even in the absence of food. 

On the other hand, we also reviewed in the previous chapter the importance of stimulation techniques to explore neural dynamics and behavior and also for possible clinical applications. In the second part of this work, we will explore Continuous-Wave Near-Infrared (CW-NIR) laser, a novel technique for neuronal dynamics modulation. This optical stimulation has a 
 large potential for effective non-invasive modulation of neuronal dynamics, but its exact mode of action is not known yet. There are different hypotheses, each one pointing to different biophysical candidates to explain the laser action  during the signal modulation, such as calcium channels, capacitance or long-term processes such as mitochondria modulation, as well as different sources for the effect, such as photo-thermal or photo-chemical elements. During this work, we will characterize the CW-NIR effect in single neurons combining experimental and theoretical approaches to narrow down the possible candidates for the source of the effect. We will also present a novel approach for closed-loop stimulation to predict spikes and stimulate at different stages of the action potential generation. 

In summary, the objectives of this thesis are:
\begin{enumerate}
    \item To explore the sequential nature of neuronal dynamics at distinct description levels. 
    \item To study sequence interval variability constraints and relationships in neural models and living circuits.
    \item To analyze the feeding CPG of \textit{Lymnaea stagnalis} to provide evidence of the universality of sequential dynamical invariants found in the pyloric CPG by:
    \begin{enumerate}
        \item Characterizing the sequential invariants cycle-by-cycle in a bursting CPG model. 
        \item Analyzing intracellular recordings with spontaneous activity and with different rhythm initiation stimulation protocols. 
    \end{enumerate}
    \item To illustrate the possible functionality of sequential dynamical invariants in biohybrid robotics. 
    \item To test the capability of CW-NIR laser stimulation to modulate neuronal dynamics by: 
    \begin{enumerate}
        \item Characterizing the CW-NIR effect in the spike waveform dynamics. 
        \item Analyzing  the ability of this neurotechnology to change the spiking rate and the circuit dynamics. 
    \end{enumerate}
    \item To study the possible biophysical candidates underlying the CW-NIR effect in model simulations. 
    \item To design and implement a new technique for CW-NIR stimulation in closed-loop. 
\end{enumerate}

Understanding the sequential patterns at the level of CPGs and single neurons allows us to identify universal principles that can be applied to broader neural circuits and systems. The sequential dynamics explored in the first part of this thesis provides foundational insights into the intrinsic timing mechanisms that govern the coordination of sequential neural processes. By exploring novel modulatory  mechanisms such as near-infrared laser stimulation, we bridge the gap between neural dynamics and novel noninvasive neurotechnologies. Finally the activity-dependent laser illumination protocol allows a precise dissection of the laser effect on the sequential activation of  ionic-channels. 
