\chapter{Motivation and Objectives}\label{c-review}
During this work, we will analyze and explore the neuronal dynamics with a bottom-up approach, focusing on their dynamics. We saw in the state of art the importance of sequential dynamics, and how we can describe this sequentiallity at different scale, from milliseconds (ionic channels) to hours or days in terms of behaviour or cycles. It is important to study neuronal dynamics at different levels to understand the mechanisims behind information processing and the subsequent outputs. To explore sequential activation at cell levels we will use intracellular recordings of single neurons and CPG circuits of \textit{Lymnaea stagnalis} combined with bursting conductance-based models. In the case of the CPGs we will explore the possible restrictions in the intervals contained cycle-by-cycle, although this approach has not always been typical in this kind of studies \cite{nadim_inter-animal_2022}, the study of the variability and time-intervals cycle-by-cycle can unveil important characteristics of their restrictions \parencite{elices_robust_2019}. Bursting activity is a good case of study for this, since it is easier to define the first and last event to conform the time-intervals. In this work, we will discuss the universality of sequential dynamical invariants, showed in the pyloric CPG \parencite{elices_robust_2019}, exploring it in the feeding CPG of \textit{Lymnaea stagnalis} in a computational model and experimental recordings. We will explore the restrictions underlying the CPG but also the possible functional role of these restriction under different cases of activation of the CPG, e.g., feeding started by the precense of food, feeding activity started by hunger even at food absence. 

On the other hand, we also reviewed in the previous chapter the importance of stimulation techniques to explore the neural dynamics and behavior but also for possible clinical applications. In the second part of this work, we will explore a novel technique for neuronal dynamics modulation: the infrared laser. This optical stimulation has potential by effective modulation of neuronal dynamics, but its exact mode of action is not known yet. There are different candidates proposed to be altered in the process of the signal modulation, such as calcium channels, capacitance or long term process as mitochondria modulation, as well as different sources for the effect, as photo-thermal or photo-chemical effect. Thus, during this work we will characterize its effect in single neurons combining experimental and theoretical approaches to narrow down the possible candidates and the source of the effect. We will also design a novel stimulation technique for closed-loop stimulation by designing and testing a new activity-dependent protocol to predict spikes and stimulate at different stage of the action potential generation. 

\todo{completar motivación que enlace los dos capítulos }

In summary the objectives of this thesis are:
\begin{enumerate}
    \item Explore the sequential nature of neuronal dynamics
    \item Analyze the feeding CPG of \textit{Lymnaea stagnalis} to test the universality of sequential dynamical invariants.
    \begin{enumerate}
        \item Characterize the invariants cycle-by-cycle in the bursting model. 
        \item Analyze intracellular recordings with spontaneous activity and with different rhythm initiation stimulations. 
    \end{enumerate}
    \item Study the restrictions of variability in the reproduction of neural models and circuits.
    \item Discuss the possible functionallity of sequential dynamical invariants in robotics. 
    \item Test the capability of Near-infrared laser stimulation to modify neural activity. 
    \begin{enumerate}
        \item Characterization of the effect in the spike waveform dynamics. 
        \item Analysis of its stimulatory action. 
    \end{enumerate}
    \item Study the possible candidates underlying the effect in model simulation. 
    \item Design and implementation of a new technique for the stimulation in closed-loop. 
\end{enumerate}

% Here you review the state of the art relevant to your thesis proposal. The idea is to present the major ideas in the state of the art right up to, but not including, your own personal brilliant ideas.

% Critical analysis and comparisons should be made by pointing out the weakness of existing solutions and strengths of your proposal. You organize this section by idea, and not by author or by publication.

% In certain situations, a background of the underlying concepts is required for better understanding of the research problem and also to improve the flow of the thesis. This could either be made an introductory part of this section or separately written in a prior section.

% \begin{table}[!th]
% \centering
% % Use, for example, p{3.5cm} style for fixed sized columns
% % consider using vbox to ensure large text is wrapped inside a column
% \begin{tabular}{|p{3cm}|p{9cm}|}
% \hline
% \textbf{Reference Type} & \textbf{Citation}\\ \hline
% Article & \cite{iqbal2018generic}\\ \hline
% Book & \cite[p.127-133]{tayyaba20205g}\\ \hline
% InProceedings & \cite{khattak2019perception}\\ \hline
% InCollection & \cite{khattak2019perception}\\ \hline
% PhD Dissertation & \cite{khattak2019perception}\\ \hline
% Masters Thesis & \cite{khattak2019toward}\\ \hline
% Technical Report & \cite{khattak2019perception}\\ \hline
% Misc & \cite{khattak2014coap}\\ \hline
% \end{tabular}
% \caption{Citation Styles.}
% \label{t-References}
% \end{table}