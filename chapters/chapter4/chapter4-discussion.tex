%Paper discussion
In this paper we have addressed the characterization of the intervals building up the rhythmic sequence in a central pattern generator model of the feeding system of \textit{Lymnaea}. This characterization has also included the study of the presence of dynamical invariants in the form of linear relationships between the sequence intervals and the cycle period of the rhythm following the experimental results that we reported in \cite{elices_robust_2019} for the crustaean pyloric CPG. All intervals building the sequence were analyzed in terms of variability and compared to the period in a cycle-by-cycle manner obtaining linear relationships in specific cases when intervals displayed high variability. 

The results found in these recordings go along with the phenomena exposed in Elliot et al. \cite{elliott_temporal_1991} where in a recording from the spontaneous feeding rhythm in \textit{Lymnaea} buccal ganglia, correlation between the period and N3t burst duration (swallow phase) was reported. In our modeling work, we have extended this analysis by characterizing each and every interval building the sequence and relating the strong linear relationships found in specific intervals to the concept of cycle-by-cycle dynamical invariants proposed in \cite{elices_robust_2019} for the coordination of robust yet flexible CPG rhythms.
%As a complement to this linear relationships, there have been analysed and exposed in form of a box-plot figure, the variability of different intervals, not only burst duration intervals but some other intervals derived. 

In the simulations, three main cases have been addressed in the analysis: N1M-driven, SO-driven and N3t-driven stimulation. This was based on the two different possibilities of inducing variability in the model according to the results reported in \cite{vavoulis_dynamic_2007}, as well as with a third neuron stimulation (N3t) which is easily addressable in the model circuit. When variability arises by the stimulation of N1M, the results are rather similar to the ones obtained stimulating N3t, observing  only a difference between N1M and N3t variability. %***discutirmos esta frase****.
Strong linear correlations to the period, i.e.,  dynamical invariants, are present mainly in all intervals related to N3 burst duration, since these are the ones with cycle-by-cycle variability related to the period variability. 
% However, here N1M presents a little more variability than in the living circuit analaysis, all linear correlations are present, being only the ones related to N3 the ones showing linear correlation with N1M.


When the stimulation is applied on SO, more intervals presenting large variability than in the other stimulation protocols are found. This is due to the specific variability of N1M and N3t burst duration, since in this case the largest influence on the period seems to come not only from N3t but also from N1M. Hence, consequently to the variability analysis, different linear relationships are found involving all intervals related to N3t and N1M. The intervals unrelated to the period are the ones associated with N2v, which is also the least variable entity in the circuit. 


These results reproduce and extend the ones discussed in \cite{vavoulis_dynamic_2007} and \cite{elliott_temporal_1991}. It is interesting to highlight the differences in the results when the period is driven by N1M/N3t or SO. When the variability is induced in N1M/N3t, N3t is still able to inhibit the N1M neuron carrying the main variability, while SO is adapting to both. However, when SO is the neuron being stimulated, since it is connected to N1M by mutual excitation, they are both boosting each other's activity, leading to the higher variability of N1M. Here, N3t must adapt to N1M, which is harder due to the SO constant excitation. For this reason, it was necessary to inject some additional current in N3t in the simulations, which would usually be an input received from the cerebral ganglia. 

Each of the three phases of the rhythm corresponds to a specific motor action: N1 protraction, N2 rasp and N3 swallow. As it is pointed out in \cite{elliott_temporal_1991}, the motor sequence consists of a two-stroke relaxation: protraction and rasp (N1,N2), in charge of moving the radula, and swallowing (N3). In these kind of systems, it is common to find one of them fixed and the other one variable, which in our case is the swallow phase. In this context, SO stimulation can be related to sucrose stimulation \cite{benjamin_distributed_2012,kemenes_analysis_1994},
%Rose1981
%todo
this could be related with the increase in N1M variability, since in presence of food, protraction phase may become crucial for an effective food reaching. Thus, there might be now two phases showing more variability in their intervals duration. The discussed invariants can participate in the coordination of these mechanisms by imposing  variability constraints in only specific intervals of the motor sequence. The results shown in this paper also indicate that distinct constraints in the form of dynamical invariants can emerge in the same network under different behavioral contexts.



The study of cycle-by-cycle invariant relationships, found not only in experimental recordings but also in computational models as we report here, highlights the presence
of organized variability in a motor rhythm. Previous results that indicated the existence of these invariants in \textit{Lymnaea Stagnalis} when stimulating two specific neurons have been reproduced computationally using Vavoulis et al. CPG feeding model and providing a wider analysis of variability, not only in the burst duration intervals but in a complete set of intervals obtained from the relation between different neurons.% in dual recordings.

It is important to emphasize that in the CPG model variability was induced by applying a current ramp to N1M, N3t or SO neurons. Sources of intrinsic variability in circuit models can include the use of chaotic neurons \cite{elices_closed-loop_2015,elices_asymmetry_2017} or noise. However, in our experience so far, such variability is not enough to generate dynamical invariants while sustaining robust rhythms in the circuit. On the other hand, living neurons variability can be propagated to model neurons to generate dynamical invariants in hybrid circuit configurations of interacting artificial and living neurons \cite{amaducci_rthybrid_2019,reyes-sanchez_automatic_2020}.

Invariants reflect a balance between the robustness of the sequence and the flexibility to accommodate longer or shorter intervals from the network interaction. The isolated and network behavior of the feeding CPG neurons was studied in~\cite{straub_endogenous_2002}, which showed the capacity of isolated N1M neurons to generate a plateau. When neurons are connected, features such as the N2v characteristic waveform or the N3t post inhibitory rebound emerge. Our model study indicates that intrinsic dynamics, network topology and synaptic dynamics contribute to the generation of the dynamical invariants.

%Therefore, the bursting activity is in all 4 neurons conditional to the inhibition from the other neurons in the circuit. For this conditional bursting and rhythm the intrinsic properties of the neurons are important, in N1M and N2v its plateau capacity, and in N3t its PIR (post inhibitory rebound) capacity, since it is the combination of those properties along with the synapse characteristics that allow the flexibility of the system and the presence of the invariant intervals. The synapse in this circuit is affected not only by the occurrence of a single spike but by its intensity and duration.}


%Dynamical invariants have not been addressed in computational models becuase typically they do not display enough variability
%In computational models literature, the presence of these dynamical invariants had not been taken into account until now. These invariants are not easy to reach in models because of the lack of sources for flexibility in theoretical paradigms. To obtain this effect there are many elements involved, from the configuration of the model itself in terms of single neuron activity and connections to the variability in the model. 

The presence of dynamical invariants in another CPG beyond the already found in \textit{Carcinus maenas} \cite{elices_robust_2019} points out to a general phenomena that can be present in other more complex sequence generating neural systems. Moreover, the results reported in this paper indicate that such invariants can be found in computational models with sufficiently rich intrinsic and circuit dynamics.







Hybrots constitute an useful resource to study neural circuits dynamics and their relationship to behavior since they provide an artificial powerplant to implement experiments that otherwise would lack one, like those performed \textit{in vitro} or \textit{in vivo} with immobilized animals. In our particular case, we designed FLC-Hybrot to analyze how the dynamical invariants present in the \textit{Carcinus maenas} pyloric CPG could be applied for the coordination of robotic locomotion. We validated that when modulated by these temporal intervals, the oscillatory motion of the robot legs remained coordinated at all times, despite all the variations in amplitude and period that took place during the experiments. Moreover, video tracking of the hybrot movement revealed that dynamical invariants were effectively translated into the relation between the oscillation amplitude and period. This resulted in a more stable and constant movement speed for the hybrot than when coordinated by intervals that did not maintain a dynamical invariant relationship.

