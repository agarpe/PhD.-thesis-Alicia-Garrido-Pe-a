In this chapter, we studied neural sequentiality analyzing the bursting activity of the feeding CPG in \textit{L. stagnalis}. As discussed before, CPGs are ideal circuits for this kind of study due to their ability to maintain a robust sequential activation and still present a high flexibility to adapt to changes in their functional context. Also, their rhythmic bursting activity provides clearer time references to define sequence intervals than other more complex neural  circuits. 

We addressed the characterization of the intervals building up the rhythmic sequence in this system with model simulations and also with experimental intracellular recordings. We identified the presence of sequential dynamical invariants in the form of linear relationships between the sequence intervals and the cycle period of the rhythm following the experimental results reported in \textcite{elices_robust_2019} for the crustacean pyloric CPG. Most common studies usually consider averaged variability, masking the information in the relations taking place at each cycle of the sequential activations. Taking that into account, here we characterized the variability cycle-by-cycle. 

In the model, we analyzed the relations reported by \textcite{elliott_temporal_1991} in an extended characterization of each and every interval building the sequence. We also related the strong linear relationships found between specific intervals to the concept of robust sequential dynamical invariants proposed in \textcite{elices_robust_2019} for the coordination of robust yet flexible CPG rhythms. In the simulations, three main cases have been addressed for the analysis: N1M-driven, SO-driven and N3t-driven stimulation. This was based on the two different possibilities of inducing variability in the model according to the results reported in \textcite{vavoulis_dynamic_2007}, as well as with a third neuron stimulation (N3t) which is easily addressable in the model circuit. When variability arises by the stimulation of N1M, the results are rather similar to the ones obtained by stimulating N3t: Strong linear correlations to the period, i.e., dynamical invariants, are present mainly in all intervals related to N3 phase, since these are the ones whose cycle-by-cycle variability is related to the period's variability.  

We also showed the redistribution of variability that takes place when the stimulation is applied on SO. In this scenario, more intervals are found presenting large variability than in the other stimulation protocols. This is due to the specific variability of N1M and N3t burst duration, since in this case the largest influence on the period seems to come not only from N3t but also from N1M. Hence, consequently to the variability analysis, different linear relationships are found involving all intervals related to N3 and N1 phases. The intervals unrelated to the period are the ones associated with N2 phase, which is also the least variable entity in the circuit. 

These results reproduce and extend the ones discussed in \textcite{vavoulis_dynamic_2007} and \textcite{elliott_temporal_1991}. It is interesting to highlight the differences in the results when the period is driven by N1M/N3t or SO. When the variability is induced in N1M/N3t, N3t is still able to inhibit the N1M neuron carrying the main variability, while SO is adapting to both. However, when SO is the neuron being stimulated, since it is connected to N1M by mutual excitation, they are both boosting each other's activity, leading to the higher variability of N1M. Here, N3t must adapt to N1M, which is harder due to the SO constant excitation. For this reason, it was necessary to inject some additional current in N3t in the simulations, which might be in the living activity, an input received from the cerebral ganglia. 

The flexibility of the model to adapt the induced variability into the system is not an easy feature to achieve. Note that in each case only N1M, N3t or SO were stimulated by a ramp current and still the variability was displayed in the whole circuit activity, e.g., stimulating N3t produced variability in the time intervals of that neuron but also in N1M, and the stimulation of SO modulated the activity in N1M and N3t. Several characteristics in the description of this model, such as the gradual synapse and the separation between fast and slow dynamics for each neuron, might be key for reaching such flexibility. The realistic definition of the waveform shape and the interrelations between the neurons in the CPG model, along with the current injection, allowed the study of dynamical invariants. However, inducing the variability by a current ramp injection hinders the functional variability generated by the neurons by their spontaneous activity, being only possible to simulate a case where the variability was produced by a perturbation in the system. This limitation is manifested when comparing the model activity with the experimental intracellular recordings, where both spontaneous activity and induced modulation recordings showed large cycle-by-cycle variability in the time-interval durations in the sequential bursting activity. 

Each of the three phases of the rhythm corresponds to a specific motor action: N1 protraction, N2 rasp and N3 swallow. As it was pointed out in \textcite{elliott_temporal_1991}, the motor sequence consists of a two-stroke relaxation: protraction and rasp (N1,N2), in charge of moving the radula, and swallowing (N3). In these kind of systems, it is common to find one of them fixed and the other one variable, which in our case is the swallow phase. In this context, SO stimulation can be related to sucrose stimulation \parencite{benjamin_distributed_2012,kemenes_analysis_1994},
this could be related with the increase in N1M variability, since in presence of food, protraction phase may become crucial for an effective food reaching. Thus, there might be now two phases showing more variability in their intervals duration. The discussed invariants can participate in the coordination of these mechanisms by imposing variability constraints in only specific intervals of the motor sequence. The results shown in this work also indicate that distinct constraints in the form of sequential dynamical invariants can emerge in the same network under different behavioral contexts.

In the experimental study, we could prove the model prediction that depending on the source of the activation, the distribution of variability and the intervals presenting these strong relationships changed. Also, even under scenarios of high variability, we showed that there were strong linear relationships in the form of sequential robust dynamical invariants between some of the sequence intervals. In the case of spontaneous activity, the most robust example was a strong linear correlation for N3 phase in a recording where the N3 seemed to be inducing periods of silence. In other two cases, the variability was shifted, with N1 having a larger relation to the period cycle-by-cycle and with a distribution of variability between both N1 and N3 phases. What we saw in this case, and in the rest of the experiments in that section, was the absence of relation between the N2 phase and the period, which means that the variability in each cycle is not carried by the rasp phase, which has a constant and low variable activation. In addition to the spontaneous recordings, we analyzed three cases of CPG activity modulation: SO neuron driven, MLN stimulation and CV1a neuron driven. Each case is associated to a different source for the activity change in relation to functional processes. For each one, we observed a different distribution of the variability between the time-intervals. During the SO modulation, in spontaneous and induced scenarios we saw, as in the model, that variability was carried by N1 and N3 phases, having both relation to the period cycle-by-cycle, being stronger in the case of N1. When stimulating the middle lip nerve (MLP), directly connected to the N1M to activate the rhythm, there was a robust sequential dynamical invariant in N1, showing that during this activation (in the simulated presence of food) the role of protraction phase was key, carrying all variability. Finally, in the CV1a neuron stimulation, the N1 phase was again the one with a larger relation to the period, showing also that a small change in that stimulation altered the distribution of variability shifting to N3 phase and the related intervals. 

With these results, we saw the potential of dynamical invariants as functional variability indicators, since they reflect a balance between the robustness of the sequence and the flexibility to accommodate longer or shorter intervals from the network interaction. The distinct sources of feeding activation changed the distribution of variability, readjusting it from one phase to another. Also, in the model we saw that this phenomenon is highly dependent on the intrinsic dynamics, network topology and synaptic dynamics that contribute to the generation of the dynamical invariants. Their observation in the model simulations where only possible when other neurons in the circuit adapted their activity to the neuron with the induced variability. 

The study of cycle-by-cycle invariant relationships, found not only in experimental recordings but also in computational models as we report here, highlights the presence of organized variability in a motor rhythm. This points to the universality of these phenomena, and the functional associations to autonomous neural coordination mechanisms. The concept of dynamical invariants can be applied to the design of novel biomarkers, which can be key for future clinical applications, specially for novel neurotechnology tools. In this line, we showed a first approach of the extrapolation of these restrictions into an effective locomotion in the FLC-Hybrot. The hexapod robot  showed that it is possible to translate the dynamical invariants into the relation between the oscillation amplitude and period, showing the possible future applications of locomotion based on dynamical CPG features. This study paves the way for further exploration of sequential dynamical invariants in more complex neural systems and their practical applications.




% It is important to emphasize that in this CPG model the variability was induced by applying a current ramp to N1M, N3t or SO neurons. Sources of intrinsic variability in circuit models can include the use of chaotic neurons \textcite{elices_closed-loop_2015,elices_asymmetry_2017} or noise. We However, in our experience so far, such variability is not enough to generate dynamical invariants while sustaining robust rhythms in the circuit. On the other hand, living neurons variability can be propagated to model neurons to generate dynamical invariants in hybrid circuit configurations of interacting artificial and living neurons \textcite{amaducci_rthybrid_2019,reyes-sanchez_automatic_2020}.

% Invariants reflect a balance between the robustness of the sequence and the flexibility to accommodate longer or shorter intervals from the network interaction. The isolated and network behavior of the feeding CPG neurons was studied in~\textcite{straub_endogenous_2002}, which showed the capacity of isolated N1M neurons to generate a plateau. When neurons are connected, features such as the N2v characteristic waveform or the N3t post inhibitory rebound emerge. Our model study indicates that intrinsic dynamics, network topology and synaptic dynamics contribute to the generation of the dynamical invariants.



% Previous results that indicated the existence of these invariants in \textit{Lymnaea Stagnalis} when stimulating two specific neurons have been reproduced computationally using Vavoulis et al. CPG feeding model and providing a wider analysis of variability, not only in the burst duration intervals but in a complete set of intervals obtained from the relation between different neurons.

% The presence of dynamical invariants in another CPG beyond the already found in \textit{Carcinus maenas} \textcite{elices_robust_2019} points out to a general phenomena that can be present in other more complex sequence generating neural systems. Moreover, the results reported in this paper indicate that such invariants can be found in computational models with sufficiently rich intrinsic and circuit dynamics.







% Hybrots constitute an useful resource to study neural circuits dynamics and their relationship to behavior since they provide an artificial powerplant to implement experiments that otherwise would lack one, like those performed \textit{in vitro} or \textit{in vivo} with immobilized animals. In our particular case, we designed FLC-Hybrot to analyze how the dynamical invariants present in the \textit{Carcinus maenas} pyloric CPG could be applied for the coordination of robotic locomotion. We validated that when modulated by these temporal intervals, the oscillatory motion of the robot legs remained coordinated at all times, despite all the variations in amplitude and period that took place during the experiments. Moreover, video tracking of the hybrot movement revealed that dynamical invariants were effectively translated into the relation between the oscillation amplitude and period. This resulted in a more stable and constant movement speed for the hybrot than when coordinated by intervals that did not maintain a dynamical invariant relationship.

