In this chapter, we studied the sequentially analyzing the bursting activity of the feeding CPG in \textit{L. stagnalis}. As discussed before, this CPG is ideal for this kind of study due to their ability to maintain a robust sequential activation and still present a high flexibility to adapt to changes in the context. Also, the bursting activity provides clearer references than in other more complex circuits. 

In this system addressed the characterization of the intervals building up the rhythmic sequence in this system with model simulations and also in experimental intracellular recordings. We tested the presence of sequential dynamical invariants in the form of linear relationships between the sequence intervals and the cycle period of the rhythm following the experimental results that we reported in \cite{elices_robust_2019} for the crustacean pyloric CPG. Most extended studies usually study the averaged variability, masking the information in the relations taking place at each cycle of sequential activations. Taking that into acount, here we characterize the variability cycle-by-cycle. 

In the model, we extended the analysis of the reported relations by \cite{Elliott1991} by characterizing each and every interval building the sequence and relating the strong linear relationships found in specific intervals to the concept of robust sequential dynamical invariants proposed in \cite{elices_robust_2019} for the coordination of robust yet flexible CPG rhythms. In the simulations, three main cases have been addressed in the analysis: N1M-driven, SO-driven and N3t-driven stimulation. This was based on the two different possibilities of inducing variability in the model according to the results reported in \cite{vavoulis_dynamic_2007}, as well as with a third neuron stimulation (N3t) which is easily addressable in the model circuit. When variability arises by the stimulation of N1M, the results are rather similar to the ones obtained stimulating N3t. Strong linear correlations to the period, i.e.,  dynamical invariants, are present mainly in all intervals related to N3 phase, since these are the ones with cycle-by-cycle variability related to the period variability.  

We also show the redistribution of variability that takes place when the stimulation is applied on SO. In this scenario more intervals presenting large variability than in the other stimulation protocols are found. This is due to the specific variability of N1M and N3t burst duration, since in this case the largest influence on the period seems to come not only from N3t but also from N1M. Hence, consequently to the variability analysis, different linear relationships are found involving all intervals related to N3 and N1 phases. The intervals unrelated to the period are the ones associated with N2 phase, which is also the least variable entity in the circuit. 

These results reproduce and extend the ones discussed in \cite{vavoulis_dynamic_2007} and \cite{Elliott1991}. It is interesting to highlight the differences in the results when the period is driven by N1M/N3t or SO. When the variability is induced in N1M/N3t, N3t is still able to inhibit the N1M neuron carrying the main variability, while SO is adapting to both. However, when SO is the neuron being stimulated, since it is connected to N1M by mutual excitation, they are both boosting each other's activity, leading to the higher variability of N1M. Here, N3t must adapt to N1M, which is harder due to the SO constant excitation. For this reason, it was necessary to inject some additional current in N3t in the simulations, which would usually be an input received from the cerebral ganglia. 

Each of the three phases of the rhythm corresponds to a specific motor action: N1 protraction, N2 rasp and N3 swallow. As it is pointed out in \cite{Elliott1991}, the motor sequence consists of a two-stroke relaxation: protraction and rasp (N1,N2), in charge of moving the radula, and swallowing (N3). In these kind of systems, it is common to find one of them fixed and the other one variable, which in our case is the swallow phase. In this context, SO stimulation can be related to sucrose stimulation \cite{benjamin_distributed_2012,kemenes_analysis_1994},
this could be related with the increase in N1M variability, since in presence of food, protraction phase may become crucial for an effective food reaching. Thus, there might be now two phases showing more variability in their intervals duration. The discussed invariants can participate in the coordination of these mechanisms by imposing  variability constraints in only specific intervals of the motor sequence. The results shown in this paper also indicate that distinct constraints in the form of dynamical invariants can emerge in the same network under different behavioral contexts.

The flexibility of this model to adapt the induced variability into the system is not an easy feature to achieve. Some characteristics in the description of this model such as the gradual synapse and the separation between fast and slow dynamics for each neuron, might be key achieving this. Although this realistic definition of the interrelations between the neurons in the CPG in this model, allowed the study of dynamical invariants, inducing the variability by a current ramp injection hinders the functional variability generated by the neurons and the spontaneous activity of the neurons. We saw this limitation when study the experimental intracellular recordings, where spontaneous activity and induced modulation recordings, showed the variability cycle-by-cycle, and the stochastic sequence of time-interval durations. 

In the experimental study, even under those scenarios of high stochasticity, we showed that there were strong linear relationships in the form of sequential robust dynamical invariants in some of the intervals. Also, as we observed in the model, depending on the source of the activation, the distribution of variability and the intervals presenting this strong relationships changed. In the case of spontaneous activity, we found a different situation for each, the most robust example was a strong linear correlation for N3 phase, in a recording where the N3 seemed to be inducing periods of silence. In the other two cases, the variability was shifted, having N1 a larger relation to the period cycle-by-cycle and with a distribution of variability between both. What we saw in this case, and in the rest of the experiments in that section, was the absence of relation between N2 phase and the period, which means, the variability in each cycle is not carried by the rasp phase, which has a constant and low variable activation. Apart from the spontaneous recording, we analyzed three cases of CPG activity modulation: SO neuron driven, MLN stimulation and CV1a neuron driven. Each case is associated to a different source or modulation of the activity, with functional processes related and for each one we observed a different distribution of the variability between the time-intervals. During the SO modulation, in spontaneous and induced scenarios we saw, as in the model, that variability was carried by N1 and N3 phases, having both relation to the period cycle-by-cycle, being stronger in the case of N1. When stimulating the middle lip nerve (MLP), directly connected to the N1M to activate the rhythm, there was a robust sequential dynamical invariant in N1, showing that during this activation (in the simulated presence of food) the role of protraction phase was key, carrying all variability. Finally in the CV1a neuron stimulation, the N1 phase was again the one with a larger relation to the period, showing also that a small change in that stimulation, altered the distribution of variability shifting to N3 phase and the related intervals. 

With these results, we saw the potential of dynamical invariants as functional variability indicators, since they reflect a balance between the robustness of the sequence and the flexibility to accommodate longer or shorter intervals from the network interaction. The distinct sources of feeding activation changed the classification of variability, adjusting the importance from one phase to another. Also, in the model we saw that this phenomena is highly dependent on the intrinsic dynamics, network topology and synaptic dynamics that contribute to the generation of the dynamical invariants. Their observation in the model simulations where only possible when other neurons in the circuit adapted their activity to the neuron with the induced variability. 

The study of cycle-by-cycle invariant relationships, found not only in experimental recordings but also in computational models as we report here, highlights the presence of organized variability in a motor rhythm. This points to the universality of this phenomena, and the functional associations to it show its potential as time indicator. The applications to this indicator can be key for future clinical applications, specially for novel neurotechnology tools. In this line, we showed a first approach of the extrapolation of this restrictions into an effective locomotion in the FLC-Hybrot. The hexapod robot  showed that it is possible to translate the dynamical invariants into the relation between the oscillation amplitude and period, showing the possible future applications of locomotion based on the CPG features. This study paves the way for further exploration of dynamical invariants in more complex neural systems and their practical applications.




% It is important to emphasize that in this CPG model the variability was induced by applying a current ramp to N1M, N3t or SO neurons. Sources of intrinsic variability in circuit models can include the use of chaotic neurons \cite{elices_closed-loop_2015,elices_asymmetry_2017} or noise. We However, in our experience so far, such variability is not enough to generate dynamical invariants while sustaining robust rhythms in the circuit. On the other hand, living neurons variability can be propagated to model neurons to generate dynamical invariants in hybrid circuit configurations of interacting artificial and living neurons \cite{amaducci_rthybrid_2019,reyes-sanchez_automatic_2020}.

% Invariants reflect a balance between the robustness of the sequence and the flexibility to accommodate longer or shorter intervals from the network interaction. The isolated and network behavior of the feeding CPG neurons was studied in~\cite{straub_endogenous_2002}, which showed the capacity of isolated N1M neurons to generate a plateau. When neurons are connected, features such as the N2v characteristic waveform or the N3t post inhibitory rebound emerge. Our model study indicates that intrinsic dynamics, network topology and synaptic dynamics contribute to the generation of the dynamical invariants.



% Previous results that indicated the existence of these invariants in \textit{Lymnaea Stagnalis} when stimulating two specific neurons have been reproduced computationally using Vavoulis et al. CPG feeding model and providing a wider analysis of variability, not only in the burst duration intervals but in a complete set of intervals obtained from the relation between different neurons.

% The presence of dynamical invariants in another CPG beyond the already found in \textit{Carcinus maenas} \cite{elices_robust_2019} points out to a general phenomena that can be present in other more complex sequence generating neural systems. Moreover, the results reported in this paper indicate that such invariants can be found in computational models with sufficiently rich intrinsic and circuit dynamics.







% Hybrots constitute an useful resource to study neural circuits dynamics and their relationship to behavior since they provide an artificial powerplant to implement experiments that otherwise would lack one, like those performed \textit{in vitro} or \textit{in vivo} with immobilized animals. In our particular case, we designed FLC-Hybrot to analyze how the dynamical invariants present in the \textit{Carcinus maenas} pyloric CPG could be applied for the coordination of robotic locomotion. We validated that when modulated by these temporal intervals, the oscillatory motion of the robot legs remained coordinated at all times, despite all the variations in amplitude and period that took place during the experiments. Moreover, video tracking of the hybrot movement revealed that dynamical invariants were effectively translated into the relation between the oscillation amplitude and period. This resulted in a more stable and constant movement speed for the hybrot than when coordinated by intervals that did not maintain a dynamical invariant relationship.

